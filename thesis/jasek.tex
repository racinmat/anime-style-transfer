\documentclass[twoside]{ctuthesis}

\ctusetup{
	mainlanguage = english,
	title-czech = {Simulace hloubkových senzorů pro autonomní učení a testování},
	title-english = {Simulating depth measuring sensors for autonomous learning and benchmarking},
	xdoctype = M,
	doctype-english = {Master thesis},
	xfaculty = F3,
	department-czech = {Katedra Kybernetiky},
	department-english = {Department of Cybernetics},
	author = {Otakar Jašek},
	supervisor = {doc. Ing. Karel Zimmermann, PhD.},
	keywords-czech = {Generativní adversarialní sítě, CycleGAN, LiDAR, hloubková data, GTA V},
	keywords-english = {Generative adversarial networks, CycleGAN, LiDAR, depth measurements, GTA V},
	fieldofstudy-english = {Open Informatics},
	subfieldofstudy-english = {Computer Vision and Digital Image},
	fieldofstudy-czech = {Otevřená Informatika},
	subfieldofstudy-czech = {Počítačové vidění a digitální obraz},
	specification-file = zadani.pdf,
	front-specification = true,
	day = 25,
	month = 5,
	year = 2018,
	ctulstbg = none,
	pkg-hyperref = true
}
\ctuprocess

\usepackage[all]{hypcap}
\usepackage[T1]{fontenc}
\usepackage[colorinlistoftodos]{todonotes}
\usepackage{gensymb}
\usepackage{subfig}
\usepackage{bm}
\usepackage{amsmath,amssymb}
\usepackage[numbers]{natbib}
\usepackage{array}

\newcolumntype{I}{>{\centering\arraybackslash} m{0.27\textwidth}}
\newcolumntype{L}{>{\centering\arraybackslash} m{0.17\textwidth}}

\begin{abstract-english}
Realistic LiDAR data that can be used for machine learning and algorithm validation and verification are hard to come by, especially in the quantities required by today's machine learning applications. Recent studies showed that it is feasible to use artificially generated images for the training of the machine learning systems focused on images. Rendering realistic RGB images has a long history driven by gaming and movie industry. Generating artificial LiDAR data are much less mature. The generated LiDAR data are often unrealistically precise and accurate.

The goal of this thesis is to introduce a pipeline which can generate LiDAR data which look as realistically as possible. We use RGB and depth data from GTA V computer game to create initial precise LiDAR representation and employed CycleGANs to introduce realism. The CycleGAN is trained with the help of real-world dataset kindly provided by the Valeo company.
\end{abstract-english}

\begin{abstract-czech}
Reálná LiDARová data využitelná pro strojové učení a verifikaci a validaci algoritmů jsou těžké získat, zejména v množství požadovaném aktuálními aplikacemi strojového učení. Nedávné studie ukazují, že je možné využít uměle vytvořené obrázky pro trénování systémů založených na strojovém učení. Uměle vygenerovaná obrazová data mají dlouhou historii zejména kvůli potřebám herního a filmového průmyslu, nicméně generování LiDARových dat je prozkoumáno mnohem méně. Uměle generovaná LiDARová data jsou také nerealisticky přesná a je třeba nejprve přidat realismus, aby bylo možné je využít pro systémy strojového učení.

Cílem této práce je vytvořit funkční algoritmus, který je schopen generovat co nejrealističtější LiDARová data. Použili jsme hloubková a RGB data z počítačové hry GTA V, abychom vytvořili přesnou LiDARovou reprezentaci a využili jsme CycleGAN k tomu, abychom tuto reprezentaci co nejvíce přiblížili reálnému světu. CycleGAN byl natrénovaný s pomocí datasetu z reálných měření, který laskavě poskytla společnost Valeo.
\end{abstract-czech}

\begin{thanks}
I would like to thank my supervisor, Karel Zimmermann, for countless advice and steering me in the right direction while working on my thesis. Also I would like to thank my girlfriend and my family for never ending support while working on this thesis.

The author of this thesis was supported by CTU Valeo Scholarship and by European Union under grant agreement No. 692455, Enable-S3.
\end{thanks}

\begin{declaration}
I declare that the presented work was developed independently and that I have listed all sources of information used within it in accordance with the methodical instructions for observing the ethical principles in the preparation of university theses.
\\\\
Signature:
\\\\
Prague, May 25, 2018
\end{declaration}

\newcommand{\descitem}[1]{\item[\sffamily\normalsize\color{ctubluetext}#1] \hfill \\}
\newcommand{\norm}[1]{\left\lVert#1\right\rVert}
\newcommand{\loss}{\mathcal{L}}
\DeclareMathOperator{\EX}{\mathbb{E}}

\begin{document}
\maketitle

\chapter{Introduction}

\section{Motivation}

A large number of machine learning applications rely on a vast amount of data from the real world to infer useful relations. However, obtaining these data is not always a viable option. For example, you would like to teach your autonomous car to recognize approaching collision in order to avoid it, but crashing real cars in order to capture the image or depth data is not possible. Luckily, computer graphics started to become more and more realistic in recent years, and it is now possible to capture image and depth data from computer games. These data have one significant advantage -- it can simulate almost any scenario such as crashing, unusual environment, etc., as long as it is possible within the game.

Although the data captured from the modern computer games look almost realistic, it suffers from many problems to be readily usable by machine learning applications. The most significant drawback is the fact, that they look {\em too} perfect -- real-world sensors often measure data with noise or fail altogether.

Our goal in this thesis is to find such a relation between the in-game data and the real world data to be able to transform the in-game data to look as realistically as possible. Since we do not have a one-to-one mapping between these data, it is necessary to apply methods of {\em unsupervised} learning. Recently, a new method suitable for unsupervised generative learning called CycleGAN emerged and it is based on Generative Adversarial Networks (GANs). This method was shown to work on various unpaired image datasets, however, as far as we know this is the first work which is trying to apply it on depth data.

\section{Thesis structure}
In this first chapter, we set up motivation and reasoning for this work and also briefly summarize contribution of this thesis. The next chapter is an overview of related theoretical work. The first section of said chapter briefly summarizes recent work in the field, while the next section explores more deeply neural networks used in this thesis. The last section of this chapter describes operation of LiDAR which we are trying to simulate in chapter \ref{experiments}.

Chapter \ref{dataset} is dedicated to the used datasets. In this chapter, we summarize key characteristics of the datasets and how they were obtained.

Chapter \ref{programs} describes all the programs written for the purpose of this thesis and shows their functionality. This chapter can also serve as a user guide for the programs.

Chapter \ref{experiments} describes the performed experiments. We also describe all the drawbacks we encountered during the experiments. The chapter ends with a showcase of results.

In the last chapter, we analyze all the results and discuss the achieved contributions of this thesis, followed by plans for the future work and conclusion.

\section{Contribution}

\todo{write contribution}

\chapter{Theory}

\section{Related work}

\section{Used neural networks} \label{nets}

The neural networks (sometimes also called artificial neural networks or ANNs) are a powerful tool of today's machine learning. The main component is an {\em artificial neuron}, a computational unit which takes an input and computes a predefined (usually linear) function with its internal parameters. This output is then optionally fed through (usually nonlinear) {\em activation function} to introduce nonlinearities in the output. The artificial neurons can be stacked next to each other to form layers, and if we connect multiple layers together, we have a neural network. We can think of the neural network as a nonlinear transformation function with multiple internal parameters.

The process of training the neural network to give the output we desire then consists of feeding the input data into the network and evaluating the performance by the {\em loss function}, which computes a real-valued number associating the actual output of the network with a notion of a "badness" in comparison with the expected output. This loss function could be for example a norm of a difference between the output of the network and the output given by a human in the case of image labeling or it could be more complex function altogether. The loss function is then minimized with respect to the internal parameters of the neural network by gradient descent algorithm. The most used gradient descent algorithm is a stochastic descent and its variants such as Adam~\cite{adam} or Adagrad~\cite{adagrad}.

If the loss function is well defined over the problem set, then the network will give the desired results at the global minimum of the loss function. However, since this function is a function of the {\em parameters} of the network, it is generally not convex and highly dimensional, and therefore it is hard to reach the global minimum. LeCun~\cite{efbackprop} gave numerous tricks to improve the likelihood of finding a good enough local minimum.

The neural layers we introduced above are usually called {\em fully connected} because all the outputs from one layer are connected to all the neurons from the next layer. This was one of the first formulations of artificial neural networks~\cite{orignet}. However, these fully connected layers are not well suited for computer vision applications, since we would like to have the same response to the particular object in the image regardless of its position or orientation. To overcome this issue, the {\em convolutional} layers~\cite{convnet}, which perform a mathematical operation of convolution over the input, were introduced. Quite often these convolutional layers are followed by the fully connected layers at the end of the network.

\subsection{Generative adversarial network (GAN)}

A generative adversarial network is a concept by Ian Goodfellow~\cite{origgan} aimed at learning to generate a sample from a particular distribution. The main goal is to train a {\em generator} network to produce samples from the target distribution given a sample from some noise distribution. In order to achieve this, a second network called {\em discriminator} is introduced, and its main goal is to distinguish between the samples from the real target distribution and "fake" samples produced by the generator network given a noise sample. If the whole setup is modeled in such a way that the trained discriminator outputs a scalar assigning a probability of the sample coming from the target distribution, the generator is then trying to produce the samples that are convincing enough to the discriminator so that discriminator's output for the generated samples is as close to 1 as possible. This setup was originally formulated within the maximum log-likelihood estimation setup. It could be seen as a two-player minimax game with the value function $V(G, D)$ shown in equation~\ref{gangame}, where $G$ and $D$ are generator and discriminator functions respectively, $p_{target}$ is a target distribution and $p_{noise}$ is a noise distribution that is usually chosen as a uniform, however it could be any other source of noise data.

\begin{equation}
\min_G \max_D V(G, D) = \EX_{\bm{x}\sim p_{target}}[\log D(\bm{x})] + \EX_{\bm{z}\sim p_{noise}} [\log(1 - D(G(\bm{z})))]
\label{gangame}
\end{equation}

This formulation immediately yields loss functions for the generator (equation \ref{ganlossg}) and discriminator (equation \ref{ganlossd}) where $\bm{x}$ are the samples from the target distribution that we present to the networks during the learning and $\bm{z}$ is a noise sampled at each iteration of the training algorithm.

\begin{equation}
\loss_G = \log(1 - D(G(\bm{z})))
\label{ganlossg}
\end{equation}

\begin{equation}
\loss_D = - (\log D(\bm{x}) + \log(1 - D(G(\bm{z}))))
\label{ganlossd}
\end{equation}

It was theoretically shown~\cite{origgan} that in the case of generator and discriminator having enough capacity this setup allows to train the generator to be able to generate samples indistinguishable from the samples from $p_{target}$. However this is not easily achievable in practice. One of the main problems is that generator usually does not have this infinite capacity. More problems stem from the fact that the original loss function (equation \ref{ganlossg}) for the generator does not provide strong enough gradients early on in the process of training, therefore a new loss function with the same theoretical properties, but stronger gradients was introduced as shown in the equation \ref{ganlossgg}.

\begin{equation}
\loss_G = -\log(D(G(\bm{z})))
\label{ganlossgg}
\end{equation}

In practice, we are trying to find the Nash equilibrium~\cite{nash} of a highly dimensional, non-convex function and while we can obtain gradients for this function using training samples, there is no known algorithm to solve this game exactly~\cite{improvedgan}. Therefore we must resort to heuristics such as optimizing discriminator near its optimum for every given sample {\em before} we optimize generator using multiple optimization steps of the discriminator if needed.

Another problem that could very easily occur is a {\em mode collapse} of the generator -- point, where generator does not use its full potential and generates a fixed point for multiple inputs keeping discriminator in the dark. Since the generator receives its gradients from the discriminator and discriminator cannot give any useful information anymore, the generator will not be updated in any sensible direction past the point where the mode collapse occurred.

\subsection{GAN variants}

Since the inception of GANs, many variants emerged trying to overcome some of the issues outlined in the previous subsection. According to DeepHunt~\cite{deephunt}, there were 354 papers proposing a variation of GAN as of 10\textsuperscript{th} May 2018. Most of these improvements revolve around redefining the loss functions and introducing various tricks to achieve better training stability.

In the following subsections will be shortly described some of these variants with their particular improvements and differences of original GAN.

\subsubsection{DCGAN -- Deep Convolutional GAN}
DCGAN~\cite{dcgan} is not a variant of GAN per se, as it mostly involves guidelines for stable training of GAN, when discriminator and generator consist of multiple convolutional layers. The said guidelines can be briefly summarized as:
\begin{itemize}
\item Use strided convolution and deconvolution instead of pooling layers. The reasoning behind this is to allow the networks to find their own representations of up sampling and down sampling.
\item Use batch normalization~\cite{batchnorm} everywhere applicable (i.e. in every layer except last). This allows to get more stable gradients for backpropagation.
\item Do not use fully connected layers that are not the direct output of the discriminator. If there are hidden fully connected layers, then the model stability might improve, however it reduces convergence speed of the training process.
\item Use ReLU~\cite{relu} activation for generator's layers except the last layer using hyperbolic tangent. It was observed, that ReLU helps convergence speed of the training process the most.
\item Use LeakyReLU~\cite{leakyrelu} activation for discriminator's layers. This seems to be especially helpful in higher resolution settings.
\item Use Adam~\cite{adam} optimizer with {\em different} hyperparameters than the usual default. Especially necessary is to lower the learning rate and momentum terms.
\end{itemize}

\subsubsection{LSGAN -- Least Squares GAN}
The main idea behind LSGAN~\cite{lsgan} is to not use maximum log-likelihood framework, but to use least squares instead. The formulation of the generator and discriminator loss functions can be then seen in equations \ref{lsganlossg} and \ref{lsganlossd}, where $a$, $b$ and $c$ are target values of the discriminator that we are aiming for. In most applications, $c = a = 1$ (or 0.9 to introduce label smoothing, as proposed by~\cite{improvedgan,smooth}) and $b = 0$.

\begin{equation}
\loss_G = \frac{1}{2}(D(G(\bm{z})) - c)^2
\label{lsganlossg}
\end{equation}

\begin{equation}
\loss_D = \frac{1}{2}(D(\bm{x}) - a)^2 + \frac{1}{2}(D(G(\bm{z})) - b)^2
\label{lsganlossd}
\end{equation}

The reasoning for this reformulation of the loss functions is mostly to provide better gradients and to move the generated samples closer to the decision boundary. In traditional GAN, samples that pass the decision boundary do not provide strong enough gradients and do not contribute to learning, however with the LSGAN, there is only one flat point of the loss functions without strong gradients.

\subsubsection{WGAN and WGAN-GP -- Wasserstein GAN (with gradient penalty)}

One of the ideas of the original GANs we have not talked about before is minimizing some metric between the generated and the target distributions. This metric is usually well defined by the respective loss function and for original formulation of GAN it is Kullback-Leibler divergence~\cite{kullback} and for LSGAN it is Pearson $\chi^2$ divergence~\cite{pearson}.

WGAN~\cite{wgan} was introduced to minimize Wasserstein-1 distance~\cite{wasser}, also known as Earth-Mover. This definition yields following loss functions as seen on equation \ref{wganlossg} and \ref{wganlossd}.

\begin{equation}
\loss_G = -D(G(\bm{z}))
\label{wganlossg}
\end{equation}

\begin{equation}
\loss_D = D(G(\bm{x})) - D(\bm{z})
\label{wganlossd}
\end{equation}

However, to enforce a Wasserstein-1 distance, it is necessary to satisfy the condition that function $D$ is K-Lipshitz continuous for any given K. This is not easy to achieve and the method used in the original paper was weight clipping to a tight bounding box after each update. It is worth noting the authors admit that this solution is impractical and obviously wrong, however they could not think of a better solution at the time.

The reformulation of WGAN called WGAN-GP~\cite{wgan-gp} emerged soon after and introduced less drastic way to enforce K-Lipshitz continuity. The discriminator's loss function would receive an additional term called gradient penalty forcing the gradients of the function to be "approximately 1 almost everywhere". This gradient penalty removes the need for the weight clipping in the original paper and it is shown in equation \ref{wgangplossgdp}, where $\bm{\tilde{x}} = \epsilon\bm{x} + (1 - \epsilon)\bm{z}$ and $\epsilon$ is a uniform random number from the range [0; 1].

\begin{equation}
\loss_{GP} = (\norm{\nabla_{\bm{\tilde{x}}}D(\bm{\tilde{x}})}_2 - 1)^2
\label{wgangplossgdp}
\end{equation}

Authors of WGAN-GP showed, that the gradient penalty is superior to the weight clipping since the original WGAN exhibited either vanishing or exploding gradients quite often. To demonstrate higher stability, WGAN-GP authors trained many architectures with this criterion on ImageNet~\cite{imagenet} dataset and measured the Inception score~\cite{improvedgan} (score based on ability to produce samples from different ImageNet class with high classification rate by Inception network~\cite{inception}) achieved by the network. It was shown~\cite{wgan-gp}, that many more architectures were able to obtain high Inception score when trained by WGAN-GP loss functions instead of original GAN loss functions.

Quite an important thing to note is the fact, that WGAN-GP {\em cannot} use popular Batch normalization~\cite{batchnorm} layers as it alters the gradients of the layers and makes the gradient penalty useless. The recommendation by the article authors is to use Layer normalization~\cite{layernorm} layers instead. It is also beneficial to widen the range of the discriminator function to [-1; 1] (as opposed to [0; 1] in the original GAN) and to achieve this, it suffices to only change the activation function of the last layer of the discriminator to hyperbolic tangent.

\subsection{SimGAN}

SimGAN~\cite{historypool} is GAN variant aiming at learning a {\em refinement} of simulated data to make them look realistic enough. The paper called the generator network with the name {\em refiner} to emphasize the fact, that it received simulated data (instead of noise) on the input. The paper introduced three important concepts in the field of GANs. The first one is {\em local} adversarial loss -- this idea means, that the discriminator should not produce only scalar output, but a response map, where each part of this map corresponds to the particular patch of the evaluated image. The reasoning behind this is to allow patches with not so dominant features to contribute to the loss of the discriminator and by extension of the refiner as well.

Second key idea is adding a so-called {\em self-regularization} term to the refiner's loss function. This term can be seen in equation \ref{simganself}, where $\bm{x}$ is a sample from simulated data (refiner's input), $\bm{\tilde{x}}$ is a refined sample (refiner's output), $\lambda_{reg}$ is a relative weight given to the importance of this loss term and $\psi$ is a feature mapping from image space to the feature space. This feature mapping could be any function with important properties (such as classification of the data), or it could be simple identity and the $\loss_{reg}$ becomes pixel-wise distance.

\begin{equation}
\loss_{reg} = \lambda_{reg}\norm{\psi({\bm{x}}) - \psi({\bm{\tilde{x}}})}_1
\label{simganself}
\end{equation}

This self-regularization term allows to learn the refiner to keep the most important information of the image during the refinement process.

Third key idea is to introduce memory for the discriminator's learning process. The discriminator should not be able to forget about the images it has seen an epoch ago. In order to facilitate this memory, a history buffer is introduced which keeps the refined samples. When performing single optimization step on discriminator weights, approximately half of the batch used in this optimization step is replaced with the samples from this history buffer. To be able to limit the buffer size, after performing the training step, half of the samples from the batch is randomly selected and replaces random samples in the history buffer.

\subsection{CycleGAN} \label{cyclegan}

CycleGAN~\cite{cyclegan} is a concept aiming at matching two different distributions by the means of GANs. The core idea is to have {\em two} GANs trained simultaneously with one generator learning the mapping from the first distribution to the second and the other generator learning the reverse mapping. To enforce this reverse mapping, new term called {\em cycle consistency loss} is added to the loss function of the generators and it can be seen on the equation \ref{cycleloss}, where $p_X, p_Y$ are the distributions between which we try to find a mapping $\bm{x}\sim p_X$, $\bm{y}\sim p_Y$, $G_{X\rightarrow Y}$ is a generator mapping from $p_X$ to $p_Y$, $G_{X\rightarrow Y}$ is a generator mapping from $p_Y$ to $p_X$ and $\lambda_{cyc}$ is a relative weight given to the importance of this loss term.

\begin{equation}
\loss_{cyc} = \lambda_{cyc}(\norm{G_{Y\rightarrow X}(G_{X\rightarrow Y}(\bm{x})) - \bm{x}}_1 + \norm{G_{X\rightarrow Y}(G_{Y\rightarrow X}(\bm{y})) - \bm{y}}_1)
\label{cycleloss}
\end{equation}

This approach of training two GANs simultaneously can give us mapping between these two distributions {\em without} having a pair-to-pair correspondences. It is important to note that even though this closely relates to the style transfer problem, the resulting mappings should work in both directions which is usually not the case with more common approaches~\cite{artstyle} to the style transfer. Since this seems like an approach that could help us model the mapping between real-life and in-game data of the cars' sensors, we decided we will use CyclGAN as a basis for our experiments using various underlying architectures of GANs.

The CycleGAN does not concern itself with the particular type of GAN used as a generator mapping, however original results were published using LSGAN with instance normalization~\cite{instancenorm} and $\lambda_{cyc} = 10$

\section{Description of LiDAR}

\chapter{Datasets} \label{dataset}

\section{Depth sensors}
Using outputs from depth sensors in neural network can be quite challenging. The main difficulty stems from the fact that even though most of the depth sensors (including LiDAR) capture data on a regular grid, there is usually needed some post-processing of the data which discards some invalid points (for various reasons, i.e. the point is too far to be considered reliable or the laser did not return any response). This post-processing usually results in a point cloud (with additional information such as intensity) of an {\em irregular} shape -- meaning there is not the same amount of points in one measurement. This is a problem that is not easily solved by neural networks with convolutional and fully-connected layers. The reasoning for why this is an obstacle is provided in section~\ref{nets}.

Since we are aiming at {\em generating} data using CycleGAN~\cite{cyclegan}, we ideally want measurement from both datasets to have equal shape. If that would not be possible for some reason, the least constraining requirement is that there is a mapping representable by a neural network which transforms a measurement from one dataset to a measurement from other dataset with matching shapes and vice versa.

To ease the work of neural networks, we decided to use representation as close to LiDAR as possible for both datasets. Velodyne HDL-64E (the LiDAR used by Valeo company) has 64 lasers (each with different vertical angle) and by default spins at 600~RPM, which according to the LiDAR manual~\cite{velomanual} means that horizontal resolution is 0.1728\degree{}. We can then create a grid of 64$\times$2084 virtual lasers, where this grid corresponds to all data points collected during one full rotation of LiDAR. The process of creating such grid consists of casting a ray from the camera center corresponding to the particular horizontal and vertical angle to the point cloud and finding the closest point to this ray. Then, threshold of the distance of the point from the ray is necessary to make sure our closest point is not too far away. We set up this threshold as 0.5~\% of the distance from the camera to simulate conic nature of the laser. This reasoning immediately shows that a multi-channel grid is necessary where at least one of the channels encodes validity of the corresponding ray. One measurement therefore consists of an "image" of size 64$\times$2084$\times$3, where first channel corresponds to the distance of the ray from the camera center, second channel corresponds to intensity of the response (information that real LiDAR outputs as well) and third channel corresponds to validity of a particular ray.

Another way a particular laser in this "image" can be invalid is if the corresponding point found in point cloud is either too far or too close from the camera center. These limits come again from the LiDAR manual, minimal distance is 0.9 meters, maximal distance is 131 meters.

It could happen, that substantial amount of information would be missing from one measurement -- especially if the point cloud was rather sparse (as it was in Valeo dataset case). Even worse, the missing information could look entirely random. To remedy this, we employed linear interpolation of the rays, that are marked as invalid and have at least half of their neighbors in a neighborhood of size 3 valid. The said interpolation involved distance and intensity as well.

There are numerous advantages of this representation. One is that such representation could be easily treated as an image by neural networks and therefore convolution is applicable. Also, for neural networks, fixed-size input is often desired. Another advantage is that this representation is easily transformable to the point cloud representation. And if we take first channel separately and mask it with validity channel, then it can be easily displayed as a depth image of size 64$\times$2048.

The only thing considering depth dataset creation we have not talked about yet is the method of obtaining the corresponding point clouds, camera center and starting rotation. Those aspects vary depending on the dataset and therefore we will talk about them in the next two subsections.

\subsection{Grand Theft Auto dataset}

\begin{figure}
\centering
\subfloat[RGB image]{{\includegraphics[width=0.48\textwidth,keepaspectratio]{img/gtargbim.png}}\label{gtargbim}}
\hfil
\subfloat[Depth image]{{\includegraphics[width=0.48\textwidth,keepaspectratio]{img/gtadepthim.png}}\label{gtadepthim}}
\caption{Example of images from GTA}
\end{figure}

Thanks to Matěj Račinský, who did tremendous work on exploiting Grand Theft Auto and extracting information from it automatically (such as depth, stencil buffer, etc.), we only had to use the data provided by his scripts. The data came in the form of the depth image such as \ref{gtadepthim} from in-game camera and corresponding camera matrix transforming the image to the world coordinates. However, due to the game limitations, it was always possible to capture only one camera at a time and it took non-zero time to switch the cameras to capture another image. Because of these limitations it took about one second of in-game time to capture the full 360\degree{} scene around the car. Data in Valeo dataset produce a full scan at a rate of 10~Hz and since we wanted to match the Valeo data as closely as possible, we simply interpolated positions of the car with 100~ms intervals. This actually created more measurements than depth images, however they are all taken from a different position in the in-game world.

All four virtual cameras sit at the height of one meter from the car center, which later proved to be too low and therefore quite a large portion of the car was reflected in LiDAR-like image. To correct for that, the center of the virtual LiDAR was shifted by 0.75 meters above the camera centers (1.75 meters above the car center).

We don't have intensity information from this LiDAR simulation, so we set intensity to all valid rays as 0.5 (maximal intensity is 1, minimal is 0). The dataset has 14046 LiDAR-like measurements, and it was split into two parts -- training and testing. Training portion of the dataset consists of 8427 measurements, testing contains 5619 measurements. Data from testing portion were not seen by the network during the training phase. The size of the dataset translates into 1404.6 seconds of in-game time that was recorded continuously.

Figure \ref{gtafullpcl} shows an example of the original pointcloud, figure \ref{gtalidarpcl} shows the recreated point cloud from the data from GTA dataset, figure \ref{gtalidardepth} shows an example of a first channel of the data. To ease the viewing, the horizontal stripe of 64$\times$2084 is cut into 4 pieces stacked on top of each other, creating the new size of 256$\times$521.

\begin{figure}
\centering
\subfloat[Full point cloud]{{\includegraphics[width=0.48\textwidth,keepaspectratio]{img/gtafullpcl.png}}\label{gtafullpcl}}
\hfil
\subfloat[Reconstructed point cloud]{{\includegraphics[width=0.48\textwidth,keepaspectratio]{img/gtalidarpcl.png}}\label{gtalidarpcl}}
\caption{GTA point clouds}
\end{figure}

\begin{figure}
\centering
\includegraphics[width=0.98\textwidth,keepaspectratio]{img/gtalidardepth.png}
\caption[First channel of LiDAR-like data from GTA dataset]{First channel of LiDAR-like data from GTA dataset. For easier viewing, the strip of data is divided into 4 equal stripes stacked on top of each other.}
\label{gtalidardepth}
\end{figure}

\subsection{Valeo dataset}
Valeo company provided us with two types of data -- raw and converted. Raw data contained UDP packets from various sensors before any processing with most prominent being Velodyne HDL-64E LiDAR and OXTS xNAV 550 which is a GNSS-aided inertial measurement system. Converted data consisted of point clouds and transformation matrices. Each point cloud corresponded to one full rotation of LiDAR and was already compensated for the movement of the car. The matrices served for transforming particular point clouds into common reference frame. This reference frame was usually the same as the coordinate frame of the first point cloud -- therefore the first point cloud had {\em identity} as this transformation matrix. The origin of the coordinate frame seemed to be in the car center -- we moved the virtual LiDAR center by two meters up to simulate it being on top of the roof of the car.

We decided it would be easier to use {\em converted} data -- mostly because it seemed that it contained precisely the same data as the raw, but without the hassle of parsing and processing Velodyne and OXTS UDP packets. Converted data also contained intensity measurements.

The dataset consists of 22 runs of lengths from 60 to 80 seconds in a cityscape only, resulting in total of 17393 full scans. Training portion of the dataset contains 10435 samples, testing part has 6958 measurements. The data were recorded from 22\textsuperscript{nd} February 2017 till 28\textsuperscript{th} March 2017 with two different cars.

Figure \ref{valeofullpcl} shows an example of the original LiDAR full scan, figure \ref{valeolidarpcl} shows the recreated point cloud from LiDAR-like data and figure \ref{valeolidardepth} shows an example of a first channel of the data. It is partitioned similarly as in figure \ref{gtalidardepth}.


\begin{figure}
\centering
\subfloat[Original LiDAR point cloud]{{\includegraphics[width=0.48\textwidth,keepaspectratio]{img/valeofullpcl.png}}\label{valeofullpcl}}
\hfil
\subfloat[Reconstructed point cloud]{{\includegraphics[width=0.48\textwidth,keepaspectratio]{img/valeolidarpcl.png}}\label{valeolidarpcl}}
\caption{Valeo point clouds}
\end{figure}

\begin{figure}
\centering
\includegraphics[width=0.98\textwidth,keepaspectratio]{img/valeolidardepth.png}
\caption[First channel of LiDAR-like data from Valeo dataset]{First channel of LiDAR-like data from Valeo dataset. For easier viewing, the strip of data is divided into 4 equal stripes stacked on top of each other.}
\label{valeolidardepth}
\end{figure}


\section{RGB sensors}

\subsection{Grand Theft Auto dataset}

\subsection{KITTI dataset}

\chapter{Programs} \label{programs}

The main program developed for the purpose of this thesis was a Python package \texttt{cycle} implementing modular CycleGAN~\cite{cyclegan} in Tensorflow~\cite{tensorflow} and two programs built on top of this package. This package and associated programs reside in a directory \texttt{mod-cycle-gan} at \url{https://gitlab.fel.cvut.cz/jasekota/master-thesis/tree/master/code/mod-cycle-gan} and will be therefore together referenced as \texttt{mod-cycle-gan}. There is also an utility program written in C++ called \texttt{dat-unpacker} which reads ADTF DAT files used by Valeo company and extracts data from them into an intermediate format similar to the one gathered from GTA. Last portion of code developed for this thesis is a simple Python module (with critical part of the code written in Cython) with simple name \texttt{data-processing}. These three programs/packages will be described more in-depth in following sections.

\section[\texttt{mod-cycle-gan}]{\texttt{mod-cycle-gan} -- Python package \texttt{cycle} and programs \texttt{train.py} and \texttt{test.py}}

Python package \texttt{cycle} is the implementation of CycleGAN~\cite{cyclegan} with large inspiration from GitHub repository of Van Huy at \url{https://github.com/vanhuyz/CycleGAN-TensorFlow}.

\subsection{Exported classes}


\begin{itemize}
\item \texttt{CycleGAN} -- Main class implementing CycleGAN.
\begin{description}
\descitem{\texttt{\_\_init\_\_()}} Constructor of this class takes numerous arguments. First two arguments (\texttt{XtoY, YtoX}) correspond to GANs to be set in cycle fashion (instances of \texttt{nets.GAN} or its subclasses), another two (\texttt{X\_feed, Y\_feed}) are for tfrecords file readers (\texttt{utils.TFReader}) and another two (\texttt{X\_name, Y\_name}) correspond to names of the dataset for pretty printing of logs and Tensorboard messages. Following argument (\texttt{cycle\_lambda}) is a $\lambda$ for cyclic loss function (for more detail see section \ref{cyclegan}). Next argument (\texttt{tb\_verbose}) is a boolean for deciding whether to create summaries for Tensorboard and following argument (\texttt{visualizer}) is a function to use for visualising the data in Tensorboard -- if this argument is set to \texttt{False} or \texttt{None} then no function will be used for visualisation.

Next four arguments (\texttt{learning\_rate, beta1, steps, decay\_from}) control optimization process -- namely initial learning rate for Adam optimizer~\cite{adam}, parameter beta1 of said optimizer, number of steps (where one step corresponds to one batch) and number of steps after which the learning rate starts to decay to eventually stop at zero. Following argument (\texttt{history}) indicates, whether to use history pool according to~\cite{historypool} and finally, last argument (\texttt{graph}) specifies the computational Tensorflow graph in which the model should be created. If it is left as \texttt{None}, then new graph will be created.
\descitem{\texttt{get\_model()}} This metod actually creates the full model in Tensorflow graph. As such, it should be only called once. It sets up all the losses and their respective optimizers. This method has no arguments.
\descitem{\texttt{train()}} This method is the main training loop. The only required argument (\texttt{checkpoints\_dir}) is the top-level checkpoints directory in which a new directory for this session is either created if needed or selected as a loading point in case of retrying training. Next two arguments (\texttt{gen\_train, dis\_train}) specify how often should be generator and discriminator trained within one training step. Next argument (\texttt{pool\_size}) specifies the size of the history pool. Following argument (\texttt{load\_model}) specifies a directory from which to load a saved model if retrying. Note that it is a path relative to top-level checkpoints directory. If this argument is \texttt{None}, new directory with current timestamp is created and new training starts. Next argument (\texttt{log\_verbose}) is a boolean specifying whether to log current loss periodically or not. Next argument (\texttt{param\_string}) specifies string which is a serialized version of arguments with which \hyperref[trainpy]{\texttt{train.py}} script was executed. This string will be saved to checkpoint directory with name \texttt{params.flagfile}. Last argument (\texttt{export\_final}) specifies whether to export final model after training as a binary protobuf used for testing.
\descitem{\texttt{export()}} This method requires two arguments -- first argument (\texttt{sess}) is a session in which a model was run and the second (\texttt{export\_dir}) is a directory in which to save the model. There will be two saved models of names \texttt{\{Xname\}2\{Yname\}.pb} and \texttt{\{Yname\}2\{Xname\}.pb} which can then be used for testing. This method is automatically at the end of the \texttt{train()} method if the last argument (\texttt{export\_final}) was set to \texttt{True}.
\descitem{\texttt{export\_from\_checkpoint()} -- static method} This method is a static counterpart of the \texttt{export()} method. It requires more arguments than method \texttt{export()} because it does not have all the book-keeping information the instance method has. First two arguments (\texttt{XtoY, YtoX}) are instances of GANs with the same model as used for training, another four arguments (\texttt{X\_normer, X\_denormer, Y\_normer, Y\_denormer}) are normalization and denormalization functions to be used for both datasets prior feeding the examples to network and converting them back to useful values. Next argument (\texttt{checkpoint\_dir}) corresponds to checkpoint directory where the model is stored and following argument (\texttt{export\_dir}) specifies directory in which the output models will be saved. Last two arguments (\texttt{X\_name, Y\_name}) correspond to the names of the datasets for easier identification of created models.
\descitem{\texttt{test\_one\_part()} -- static method} This method tests the stored exported model with a NumPy file and saves the important outputs of the network to a new NumPy file. First argument (\texttt{pb\_model}) is a path to an exported binary protobuf model of the network to test. Another argument (\texttt{infile}) specifies the path to the input file to test and next argument (\texttt{outfile}) corresponds to a path of output file. This output file will be a Npz NumPy file with three fields -- \texttt{output} (output generated by corresponding generator) \texttt{d\_input} (value of corresponding discriminator evaluated on input data) and \texttt{d\_output} (value of corresponding discriminator evaluated on output data).

Last argument (\texttt{include\_input}) is a boolean specifying whether to include input data in the output file or not. If set to \texttt{True}, outfile will become larger, however it will be more self-contained.
\end{description}
\item \texttt{utils.TFReader} -- Class for reading tfrecords file, which is a TensorFlow binary format for efficient storage of data and features based on Protobuf.
\begin{description}
\descitem{\texttt{\_\_init\_\_()}}
\descitem{\texttt{feed()}}
\end{description}
\item \texttt{utils.TFWriter} -- Class for creating tfrecords file from NumPy files.
\begin{description}
\descitem{\texttt{\_\_init\_\_()}}
\descitem{\texttt{run()}}
\end{description}
\item \texttt{utils.DataBuffer} -- Class implementing history pool according to~\cite{historypool}.
\begin{description}
\descitem{\texttt{\_\_init\_\_()}}
\descitem{\texttt{query()}}
\end{description}
\item \texttt{nets.BaseNet} -- Base class for mapping networks (Generator and Discriminator).
\begin{description}
\descitem{\texttt{\_\_init\_\_()}}
\descitem{\texttt{transform()}}
\descitem{\texttt{weight\_loss()}}
\end{description}
\item \texttt{nets.GAN} -- Implementation of GAN~\cite{origgan}. Uses original loss functions.
\begin{description}
\descitem{\texttt{\_\_init\_\_()}}
\descitem{\texttt{gen\_loss()}}
\descitem{\texttt{dis\_loss()}}
\end{description}
\item \texttt{nets.LSGAN} -- Implementation of Least Squares GAN~\cite{lsgan}. Subclass of \texttt{nets.GAN}.
\begin{description}
\descitem{\texttt{\_\_init\_\_()}}
\descitem{\texttt{gen\_loss()}}
\descitem{\texttt{dis\_loss()}}
\end{description}
\item \texttt{nets.WGAN} -- Implementation of Wasserstein GAN with gradient penalty~\cite{wgan}. Subclass of \texttt{nets.GAN}.
\begin{description}
\descitem{\texttt{\_\_init\_\_()}}
\descitem{\texttt{gen\_loss()}}
\descitem{\texttt{dis\_loss()}}
\descitem{\texttt{grad\_loss()}}
\descitem{\texttt{full\_dis\_loss()}}
\end{description}
\end{itemize}

\section[\texttt{dat-unpacker}]{\texttt{dat-unpacker} -- C++ utility}

\section[\texttt{data-processing}]{\texttt{data-processing} -- Python package}

\chapter{Experiments}

In this chapter we will show evaluation of two different network architectures, VGG16~\cite{vgg16} and ZFNet~\cite{zfnet}. The datasets used for evaluation were our synthesised victims dataset and KITTI~\cite{kitti} object detection dataset. All predictions were marked as matched, if the overlap of predicted region and ground truth region had intersection over union (IoU) at least 50~\% (argument \pyarg{over} of tool \hyperref[pr]{\texttt{precision\_recall.py}} set to 0.5).

Data from our experiments can be found at \url{https://gitlab.fel.cvut.cz/jasekota/jasek-thesis-data}.

\section{Datasets}
Both datasets contain subsets to use for training and validation and Victims dataset also contains usable testing subset.

\subsection{Victims dataset}
Victims dataset is a dataset that artificially merges real data from various sources -- all images were generated by letting random 3D models of humans "fall" into the scene as if they were victims of a crime or a natural disaster. Therefore this dataset proves to be quite challenging since the position and appearance of object to be detected is deformed compared to standard datasets.

The whole dataset contains 4986 images and is split into training, validation and testing subset which all contain approximate third of images -- training subset contains 1604 images, validation subset contains 1645 images and testing subset contains 1737 images. Virtually all of these images contain only one object of class person -- there is 4989 objects in the whole dataset and 4986 images. For obvious reasons, all of them are only positive examples.

This dataset did not have any ground truth files, instead it used only mask above each image to depict the sought object. Therefore it is needed to first create ground truth files by using \hyperref[bbox]{\texttt{victims\_bbox.py}} tool.

Example of images from Victims dataset can be seen at figure \ref{victex}

\begin{figure}[!h]
\begin{subfigure}{.5\textwidth}
\centering
\includegraphics[width=0.98\linewidth,keepaspectratio]{img/vict1.png}
\end{subfigure}%
\begin{subfigure}{.5\textwidth}
\centering
\includegraphics[width=0.98\linewidth,keepaspectratio]{img/vict2.png}
\end{subfigure}
\caption{Example of images from Victims dataset}
\label{victex}
\end{figure}

\subsection{KITTI dataset}

KITTI~\cite{kitti} dataset is a standard dataset used for evaluating object detection and recognition algorithms in automated driving. The dataset is split into training and testing subset by its authors, however only ground truth files for training subset are accessible by public. The ground truth files for testing subset are private and you can evaluate your algorithm on testing subset by submission of your results on KITTI server. This methodology was not suitable for our evaluation since we re-trained our networks on many different settings, so therefore we ignored testing subset and split training subset into training and validation portions in approximate ratio of 7:3. Training portion contained 5267 images and validation portion contained 2214 images totalling in 7481 images and we re-trained our networks only on training portion.

This dataset contains three different classes -- Car, Pedestrian and Cyclist. Class Car has 20160 instances in training portion and 8582 in validation portion totalling in 28742 instances throughout the whole original training subset. Class Pedestrian has 3298 instances in training portion and 1189 in validation portion totalling in 4487 instances throughout the whole original training subset. Class Cyclist has 1174 instances in training portion and 453 instances in validation portion totalling in 1627 instances throughout the whole original training subset. We can see that class Car has 17.7~times more instances than class Cyclist and about 6.4~times more instances than class Pedestrian. This ratio is more or less consistent within both portions of original training dataset.

Examples of images from KITTI dataset can be seen on figure \ref{kittiex}

\begin{figure}[!h]
\begin{subfigure}{.5\textwidth}
\centering
\includegraphics[width=0.98\linewidth,keepaspectratio]{img/kitti1.png}
\end{subfigure}%
\begin{subfigure}{.5\textwidth}
\centering
\includegraphics[width=0.98\linewidth,keepaspectratio]{img/kitti2.png}
\end{subfigure}
\caption{Example of images from KITTI dataset}
\label{kittiex}
\end{figure}

\section{Evaluation of original networks} \label{orig}

We show evaluation of networks that were trained only on VOC2007~\cite{voc2007} dataset as obtained from original Faster \rcnn{} paper and therefore the evaluation will be performed on whole datasets.

\subsection{Victims dataset}
This dataset is challenging, since the only correctly classified class was person and in a lot of the images, the person is deformed into victim position which generally does not resemble normal human position in which you would expect a person to be. Because of these deformations, the network is not performing as well as one might expect. Most probable reason for expected decrease in performance is the fact, that dataset actually used for training these networks (VOC2007~\cite{voc2007} training and validation datasets) did not contain many examples of people appearing in these unnatural positions.

On figure \ref{vicper} we can see total performance for all three subsets combined into the whole dataset evaluated by VGG16 network and ZFNet respectively. It is clearly visible, that performance of VGG16 is considerably higher than the performance of ZFNet. This is compensated by shorter runtime for ZFNet, as shown in \ref{time}, but the runtime improvement is not that significant to justify such a drastic decrease in performance.

\begin{figure}[!]
\includegraphics[width=0.7\textwidth,keepaspectratio]{img/both/vict_cmp.eps} 
\caption{Performance on whole victims dataset}
\label{vicper}
\end{figure}

What might be of concern is the fact that none of these architectures were able to achieve perfect recall. This is due to the way Faster \rcnn{} network operates -- it selects only 300 best predictions (and even then some of them have confidence close to 0) and therefore some of the predictions that would amount to higher recall (with obvious loss of precision) are not selected by the network.

\subsection{KITTI dataset}
Figure \ref{kittio} shows precision-recall curves for whole training dataset evaluated by VGG16 and ZFNet networks respectively. Ground truth files were obtained by \hyperref[ktx]{\texttt{kitti\_to\_xml.py}} tool.

\begin{figure}[!]
\includegraphics[width=0.7\textwidth, keepaspectratio]{img/both/kitti_cmp.eps}
\caption{Performance on KITTI dataset measured on all three classes}
\label{kittio}
\end{figure}

Once again you can see much better performance by VGG16 architecture than by ZFNet. Same as on victims dataset, we were not able to achieve perfect recall, even worse, our maximal recall values were even lower. It can be seen that precision starts to drop rapidly in both networks after hitting recall of about 0.3. One can argue that this implies, that KITTI dataset is even harder than victims dataset -- once again though, those values are on networks that were not specifically trained for KITTI dataset. Given this, the results are not as bad as they might seem on the first sight.

Figure \ref{kittic} depicts performance evaluation on whole KITTI training dataset being broken down by each class. It is visible, that class Car was the most succesful one. What is however quite alarming is almost complete absence of correctly classified class Cyclist. One can attribute it to almost no training examples of class \pyarg{bike} in VOC2007 training dataset. The rule of VGG16 net outperforming ZFNEt holds in every class except class Cyclist.

\begin{figure}[!]
\includegraphics[width=0.7\textwidth, keepaspectratio]{img/both/kitti_cls_cmp.eps}
\caption{Performance on KITTI dataset broken down by different classes}
\label{kittic}
\end{figure}

\section{Evaluation of re-trained networks} \label{our}

In this section we will evaluate re-trained networks specifically trained for concrete tasks. Each network was trained 5~times with different amount of training iterations by Caffe framework -- 2000, 4000, 6000, 8000 and 10000 iterations. Evaluation was then performed only on the portion of dataset that was not present in re-training phase. For KITTI dataset it was the validation portion of the training subset consisting of 2214 images, for Victims dataset it was the testing and validation subsets put together consisting of 3382 images overall.

\subsection{VGG16 network re-trained on KITTI dataset}

Figure \ref{vggkittiall} depicts performance of VGG16 network re-trained on KITTI dataset evaluated for all three classes together. It can be seen that re-training of VGG16 network increased performance as expected. Increasing number of iterations of Caffe solver performing re-training kept improving the performance of a network, however with each increase of number of iterations, the increase in performance was lower. Quite interestingly, re-training with 6000 iterations performed better than 8000 iterations. It seems that limit for recall on Faster~\rcnn{} network is close to value of 0.75.

\begin{figure}[!]
\includegraphics[width=0.7\textwidth, keepaspectratio]{img/vgg/vgg_kitti_all.eps}
\caption[Performance of VGG16 network on KITTI dataset]{Performance on KITTI dataset of VGG16 network re-trained on KITTI dataset measured on all three classes}
\label{vggkittiall}
\end{figure}

Figures \ref{vggkittic}, \ref{vggkittip} and \ref{vggkittib} show performances of such re-trained network on classes Car, Pedestrian and Cyclist respectively. Class Car is significantly outperforming both other classes -- this can be possibly attributed to much higher number of instances of class Car in training dataset. But since there are so many more objects of class Car, overall performance as seen on figure \ref{vggkittiall} is just slightly worse than the performance of class Car - it simply outweights the worse performance of other classes.


\begin{figure}[!]
\includegraphics[width=0.7\textwidth, keepaspectratio]{img/vgg/vgg_kitti_car.eps}
\caption[Performance of VGG16 network on KITTI dataset, class Car]{Performance on KITTI dataset of VGG16 network re-trained on KITTI dataset measured on class Car}
\label{vggkittic}
\end{figure}

\begin{figure}[!]
\includegraphics[width=0.7\textwidth, keepaspectratio]{img/vgg/vgg_kitti_person.eps}
\caption[Performance of VGG16 network on KITTI dataset, class Pedestrian]{Performance on KITTI dataset of VGG16 network re-trained on KITTI dataset measured on class Pedestrian}
\label{vggkittip}
\end{figure}

\begin{figure}[!]
\includegraphics[width=0.7\textwidth, keepaspectratio]{img/vgg/vgg_kitti_bike.eps}
\caption[Performance of VGG16 network on KITTI dataset, class Cyclist]{Performance on KITTI dataset of VGG16 network re-trained on KITTI dataset measured on class Cyclist}
\label{vggkittib}
\end{figure}

Another observation is the large improvement of class Cyclist over untrained network. For the network with 10000 iterations, it even outperformed class Pedestrian which has 2.81~times more objects in training dataset.

\subsection{VGG16 network re-trained on Victims dataset}

Figure \ref{vggvictall} shows performance of VGG16 network re-trained on Victims dataset. Performance rapidly increased to precision values being above around 0.95 for recall values of 0.9 for networks re-trained by 8000 and 10000 iterations. However it seems that this might be maximal achievable performance since performance for 8000 and 10000 iterations are nearly identical. Much higher performance on Victims dataset than the performance on KITTI dataset is most likely to be attributed to the fact that Victims dataset is a lot more consistent than KITTI dataset therefore having training subset more closely related to the testing and validation subsets. However, another probable cause of such high performance might be overfitting on our dataset. To investigate further this cause we would need independent dataset that is closely related to Victims dataset but comes from a different domain.

\begin{figure}[!]
\includegraphics[width=0.7\textwidth, keepaspectratio]{img/vgg/vgg_vict_all.eps}
\caption[Performance of VGG16 network on Victims dataset]{Performance on Victims dataset of VGG16 network re-trained on Victims dataset}
\label{vggvictall}
\end{figure}

\subsection{VGG16 network re-trained on KITTI and Victims datasets}

In another experiment we were re-training the networks on both training datasets together. Since such network was re-trained on both datasets, we can evaluate them on both datasets as well.

\subsubsection{Evaluation of KITTI dataset by VGG16 network re-trained on both datasets}

Figure \ref{vggkittivall} depicts performance of VGG16 network re-trained on both datasets evaluated on all three classes of KITTI dataset. We can see that while the increasing tendence in performance is fairly similar to such in network re-trained only on KITTI dataset, amount of the increase is lower, most notably in network re-trained only by 2000 iterations. Figures \ref{vggkittivc}, \ref{vggkittivp} and \ref{vggkittivb} show performance on classes Car, Pedestrian and Cyclist respectively. One would assume that the most decrease in comparison to network re-trained only on KITTI dataset would occur in class Pedestrian since Victims dataset contains only occurences of class Person which are from completely different settings, however the most decrease is in class Cyclist -- we attribute this to even higher underrepresentation of class Cyclist in training set.

\begin{figure}[!]
\includegraphics[width=0.7\textwidth, keepaspectratio]{img/vgg/vgg_kitti_vict_all.eps}
\caption[Performance of VGG16 network on KITTI+Victims datasets]{Performance on KITTI dataset of VGG16 network re-trained on both datasets measured on all three classes}
\label{vggkittivall}
\end{figure}

\begin{figure}[!]
\includegraphics[width=0.7\textwidth, keepaspectratio]{img/vgg/vgg_kitti_vict_car.eps}
\caption[Performance of VGG16 network on KITTI+Victims datasets, class Car]{Performance on KITTI dataset of VGG16 network re-trained on both datasets measured on class Car}
\label{vggkittivc}
\end{figure}

\begin{figure}[!]
\includegraphics[width=0.7\textwidth, keepaspectratio]{img/vgg/vgg_kitti_vict_person.eps}
\caption[Performance of VGG16 network on KITTI+Victims datasets, class Pedestrian]{Performance on KITTI dataset of VGG16 network re-trained on both datasets measured on class Pedestrian}
\label{vggkittivp}
\end{figure}

\begin{figure}[!]
\includegraphics[width=0.7\textwidth, keepaspectratio]{img/vgg/vgg_kitti_vict_bike.eps}
\caption[Performance of VGG16 network on KITTI+Victims datasets, class Cyclist]{Performance on KITTI dataset of VGG16 network re-trained on both datasets measured on class Cyclist}
\label{vggkittivb}
\end{figure}

Figure \ref{vggkittivcmp} shows comparison of performances of network re-trained by 10000 iterations only on KITTI dataset and on both datasets so we can clearly see that although the perfomance is better if network is re-trained for one task only, the decrease in performance is not that harsh and it might be actually suitable to train such network on multiple dataset in order to use only one network for recognition of multiple different datasets afterwards.

\begin{figure}[!]
\includegraphics[width=0.7\textwidth, keepaspectratio]{img/vgg/vgg_kitti_vict_cmp.eps}
\caption[Comparison of performances of VGG16 networks, KITTI dataset]{Comparison of performances on KITTI dataset of VGG16 network re-trained on KITTI dataset and on both datasets}
\label{vggkittivcmp}
\end{figure}

\subsubsection{Evaluation of Victims dataset by VGG16 network re-trained on both datasets}
The situation with Victims dataset is quite similar to the one of KITTI dataset. Figure \ref{vggvkall} shows performance of VGG16 network re-trained on both datasets evaluated on Victims dataset. Once again it is lower than performance of such network re-trained only on Victims dataset however performance is still quite considerably high. Figure \ref{vggvkcmp} shows similar comparison as figure \ref{vggkittivcmp} only for Victims dataset.

\begin{figure}[!]
\includegraphics[width=0.7\textwidth, keepaspectratio]{img/vgg/vgg_vict_kitti_all.eps}
\caption[Performance of VGG16 network on Victims+KITTI datasets]{Performance on Victims dataset of VGG16 network re-trained on both datasets}
\label{vggvkall}
\end{figure}

\begin{figure}[!]
\includegraphics[width=0.7\textwidth, keepaspectratio]{img/vgg/vgg_vict_kitti_cmp.eps}
\caption[Comparison of performances of VGG16 networks, Victims dataset]{Comparison of performances on Victims dataset of VGG16 network re-trained on KITTI dataset and on both datasets}
\label{vggvkcmp}
\end{figure}

\begin{figure}[!]
\includegraphics[width=0.7\textwidth, keepaspectratio]{img/zf/zf_kitti_all.eps}
\caption[Performance of ZFNet network on KITTI dataset]{Performance on KITTI dataset of ZFNet network re-trained on KITTI dataset measured on all three classes}
\label{zfkittiall}
\end{figure}
\clearpage

\subsection{ZFNet network re-trained on KITTI dataset}

ZFNet performed quite badly on re-training overall. It seems that training method (described in \ref{train}) is not suitable for ZFNet since it actually performed worse than the original network trained only on VOC2007~\cite{voc2007} trainval. Another reason for such decrease might be in incorrect learning parameters of a network however the parameters were the same as used in original Faster \rcnn~\cite{faster} paper.

Figure \ref{zfkittiall} shows performance of ZFNet network re-trained on KITTI dataset. As you can see, the performance actually decreased and number of iterations (at least in the range of 2000 - 10000) had virtually no effect on amount of decrease. If you break down the previous figure by measured class, we get figures \ref{zfkittic} for class Car, \ref{zfkittip} for class Pedestrian and \ref{zfkittib} for class Cyclist. What is quite interesting is the fact, that performance for class Cyclist actually increased but only because it was already quite bad at the beginning.


\begin{figure}[!]
\includegraphics[width=0.7\textwidth, keepaspectratio]{img/zf/zf_kitti_car.eps}
\caption[Performance of ZFNet network on KITTI dataset, class Car]{Performance on KITTI dataset of ZFNet network re-trained on KITTI dataset measured on class Car}
\label{zfkittic}
\end{figure}

\begin{figure}[!]
\includegraphics[width=0.7\textwidth, keepaspectratio]{img/zf/zf_kitti_person.eps}
\caption[Performance of ZFNet network on KITTI dataset, class Pedestrian]{Performance on KITTI dataset of ZFNet network re-trained on KITTI dataset measured on class Pedestrian}
\label{zfkittip}
\end{figure}

\begin{figure}[!]
\includegraphics[width=0.7\textwidth, keepaspectratio]{img/zf/zf_kitti_bike.eps}
\caption[Performance of ZFNet network on KITTI dataset, class Cyclist]{Performance on KITTI dataset of ZFNet network re-trained on KITTI dataset measured on class Cyclist}
\label{zfkittib}
\end{figure}

\begin{figure}[!]
\includegraphics[width=0.7\textwidth, keepaspectratio]{img/zf/zf_vict_all.eps}
\caption[Performance of ZFNet network on Victims dataset]{Performance on Victims dataset of ZFNet network re-trained on Victims dataset}
\label{zfvictall}
\end{figure}

\subsection{ZFNet network re-trained on Victims dataset}
Figure \ref{zfvictall} shows performance of ZFNet network re-trained on Victims dataset. Surprisingly, although it preformed quite badly on KITTI dataset, the network actually improved on Victims dataset, though not by much. The performance still remains below the performance of untrained VGG16 network. Quite interesting is the fact, that the ZFNet network re-trained by 4000 iterations performed worse than the original ZFNet network while all the iterations outperform it.

\subsection{ZFNet network re-trained on KITTI and Victims datasets}
\subsubsection{Evaluation of KITTI dataset by ZFNet network re-trained on both datasets}
Figure \ref{zfkittivall} depicts performance of ZFNet network re-trained on both datasets evaluated on all three classes of KITTI dataset. Once again we can see that ZFNet network is not suitable for re-training with our parameters and the results are as bad as the results of the network re-trained only on KITTI dataset. Figures \ref{zfkittivc}, \ref{zfkittivp} and \ref{zfkittivb} show performance on classes Car, Pedestrian and Cyclist respectively. The performance is about the same as for the network fine.tuned only on KITTI dataset. Figure \ref{zfkittivcmp} shows comparison of the performance of the network re-trained by 10000 iterations only on KITTI dataset and on both datasets. We can see that network re-trained only on KITTI dataset was slightly worse, but the difference between those two is marginal.

\begin{figure}[!]
\includegraphics[width=0.7\textwidth, keepaspectratio]{img/zf/zf_kitti_vict_all.eps}
\caption[Performance of ZFNet network on KITTI+Victims dataset]{Performance on KITTI dataset of ZFNet network re-trained on both datasets measured on all three classes}
\label{zfkittivall}
\end{figure}

\begin{figure}[!]
\includegraphics[width=0.7\textwidth, keepaspectratio]{img/zf/zf_kitti_vict_car.eps}
\caption[Performance of ZFNet network on KITTI+Victims dataset, class Car]{Performance on KITTI dataset of ZFNet network re-trained on both datasets measured on class Car}
\label{zfkittivc}
\end{figure}

\begin{figure}[!]
\includegraphics[width=0.7\textwidth, keepaspectratio]{img/zf/zf_kitti_vict_person.eps}
\caption[Performance of ZFNet network on KITTI+Victims dataset, class Pedestrian]{Performance on KITTI dataset of ZFNet network re-trained on both datasets measured on class Pedestrian}
\label{zfkittivp}
\end{figure}

\begin{figure}[!]
\includegraphics[width=0.7\textwidth, keepaspectratio]{img/zf/zf_kitti_vict_bike.eps}
\caption[Performance of ZFNet network on KITTI+Victims dataset, class Cyclist]{Performance on KITTI dataset of ZFNet network re-trained on both datasets measured on class Cyclist}
\label{zfkittivb}
\end{figure}
\clearpage
\begin{figure}[!]
\includegraphics[width=0.7\textwidth, keepaspectratio]{img/zf/zf_kitti_vict_cmp.eps}
\caption[Comparison of performances of ZFNet networks, KITTI dataset]{Comparison of performances on KITTI dataset of ZFNet network re-trained on KITTI dataset and on both datasets}
\label{zfkittivcmp}
\end{figure}
\subsubsection{Evaluation of Victims dataset by ZFNet network re-trained on both datasets}

\begin{figure}[!]
\includegraphics[width=0.7\textwidth, keepaspectratio]{img/zf/zf_vict_kitti_all.eps}
\caption[Performance of ZFNet network on Victims+KITTI dataset]{Performance on Victims dataset of ZFNet network re-trained on both datasets}
\label{zfvkall}
\end{figure}

\begin{figure}[!]
\includegraphics[width=0.7\textwidth, keepaspectratio]{img/zf/zf_vict_kitti_cmp.eps}
\caption[Comparison of performances of ZFNet networks, Victims dataset]{Comparison of performances on Victims dataset of ZFNet network re-trained on KITTI dataset and on both datasets}
\label{zfvkcmp}
\end{figure}

The tendence that Victims dataset is easier for re-training holds even for network re-trained on both datasets. However it is still not able to achieve performance of original VGG16 network, not to mention performance of VGG16 network re-trained specifically for Victims dataset. Figure \ref{zfvkall} shows performance of such network. Figure \ref{zfvkcmp} compares performances of ZFNet networks re-trained on Victims dataset only and on Victims and KITTI dataset. We can see, that training specifically for particular dataset only, yields better resuls even for ZFNet network.

\clearpage
\section{Time evaluation} \label{time}

Due to hardware limitations on the server where we were running the experiments and unexpected change of hardware during experiments (upgrade from NVidia GeForce GTX Titan Black to NVidia Tesla K40c) we were unable to ensure same conditions for all experiments required for fair comparison of timing. However since one experiment consisted of re-training a network and running tool \hyperref[rec]{\pyarg{recognize.py}} on all datasets we have (including configurations such as re-training on KITTI dataset and recognizing of Victims dataset which makes no sense to later evaluate performance on, but is easier to automate), and since each GPU had the same amount of such experiments, it is actually feasible to compare timing for recognition portion. We do not have any timing data for re-training portion of the experiments, however by experience, re-training ZFNet network was about 3 times faster than re-training VGG16 network.

Table \ref{timebd} shows average times for both datasets expressed in miliseconds needed to process one image. The times could be most likely a bit faster since more comupting jobs were running at GPUs concurrently, however the ratio in between those two architecture would still most likely stay the same, showing that recognizing an image by ZFNet is about 2.4 times faster than by VGG16 network.

Table \ref{timedd} shows average times dependant on dataset being currently recognized. We can see that KITTI dataset was slightly faster than Victims dataset, however the difference is a lot more significant for VGG16 network than for ZFNet.

Tables \ref{timec1} and \ref{timec2} shows runtime performance of original networks measured on CPU (Intel\circledR{ }Xeon\circledR~{ }E5-2630 v3). VGG16 architecture proved again to be about 2 times slower than ZFNet. Runtime performance on 8-core CPU is about 11.7 times slower than running on GPUs. KITTI dataset recognition was again faster than recognition of Victims dataset, on CPU a lot more notably than on GPU.

\begin{table}[!h]
\begin{tabular}{r|ccc}
 & Average time [ms] & Min time [ms] & Max time [ms] \\
\hline
All experiments & 252.61 & 60.17 & 1459.43 \\
\textbf{VGG16} architecture & 350.14 & 149.78 & 1459.43 \\
\textbf{ZFNet} architecture & 141.09 & 60.17 & 1382.91
\end{tabular}
\caption{Time comparison of different architectures for both datasets measured on GPU}
\label{timebd}
\end{table}

\begin{table}[!h]
\begin{tabular}{r|ccc}
 & Average time [ms] & Min time [ms] & Max time [ms] \\
\hline
\multicolumn{4}{c}{KITTI dataset} \\ \hline
Both architectures & 228.50 & 68.62 & 1035.76 \\
\textbf{VGG16} architecture & 316.86 & 149.78 & 1035.76 \\
\textbf{ZFNet} architecture & 140.14 & 68.62 & 493.94 \\
\hline
\multicolumn{4}{c}{Victims dataset} \\ \hline
Both architectures & 271.29 & 60.17 & 1459.43 \\
\textbf{VGG16} architecture & 400.07 & 189.22 & 1459.43 \\
\textbf{ZFNet} architecture & 142.51 & 60.17 & 1382.91 \\
\end{tabular}
\caption{Time comparison of different architectures and different datasets measured on GPU}
\label{timedd}
\end{table}

\begin{table}[!h]
\begin{tabular}{r|ccc}
 & Average time [ms] & Min time [ms] & Max time [ms] \\
\hline
All experiments & 2971.88 & 1454.86 & 7958.40 \\
\textbf{VGG16} architecture & 3944.13 & 2775.03 & 7958.40 \\
\textbf{ZFNet} architecture & 1999.63 & 1454.86 & 5359.79
\end{tabular}
\caption{Time comparison of different architectures for both datasets measured on CPU}
\label{timec1}
\end{table}

\begin{table}[!h]
\begin{tabular}{r|ccc}
 & Average time [ms] & Min time [ms] & Max time [ms] \\
\hline
\multicolumn{4}{c}{KITTI dataset} \\ \hline
Both architectures & 2627.41 & 1454.86 & 7958.40 \\
\textbf{VGG16} architecture & 3525.56 & 2775.03 & 7958.40 \\
\textbf{ZFNet} architecture & 1729.26 & 1454.86 & 2740.61 \\
\hline
\multicolumn{4}{c}{Victims dataset} \\ \hline
Both architectures & 3488.72 & 1804.07 & 6373.09 \\
\textbf{VGG16} architecture & 4572.15 & 3702.96 & 6373.09 \\
\textbf{ZFNet} architecture & 2405.30 & 1804.07 & 5359.79 \\
\end{tabular}
\caption{Time comparison of different architectures and different datasets measured on CPU}
\label{timec2}
\end{table}

\chapter{Conclusion}

\section{Discussion of the achieved results}

It is important to note, that there seems to be no universally accepted evaluation metric of realism. Most of today's generative networks are compared using Inception score~\cite{inception}, however, this is only suitable if one is generating actual images and not the depth image with completely different semantics. Another option researches often use is Amazon's Mechanical Turk, however this suffers from the limitation that it might not be clearly obvious how a ``real'' LiDAR-like image (or reconstructed point cloud) from a GTA or Valeo dataset should look like. However, if we look at the data shown in the figures \ref{evalcmpg2v} and \ref{evalcmpv2g}, we can safely say, that the original GAN does not work in this settings regardless of the self regularization of the generators.

This seems to hold for LSGAN without self-regularization as well, however when self-regularization was added, LSGAN started to perform a lot better, preserving the important intensity information. However, it is safe to say that even without rigorous evaluation metric, WGAN-GP with self-regularization term was the winner. It seems to be able to maintain the important information and to introduce noise when transforming from the GTA dataset to Valeo dataset, as seen in the point cloud portion of the results. \todo{probably not finished yet}

\section{Future work}

In the future we would like to develop a sensible metric for generating depth data. This can be seen as the biggest shortcoming of this thesis. It will be also beneficial to leverage more information from the data in order to be able to create more accurate models of the depth sensors. The most beneficial would be to use RGB data with the LiDAR data simultaneously, however this was not possible due to the missing calibrated RGB images in the Valeo dataset.

It might be interesting to explore the idea of asymmetric CycleGAN in which the generators' structures as well as the input and output shapes differ. This use-case can be potentially very interesting as it can lead to using more information from one dataset if available.

\section{Conclusion}

\todo{todo}


\newpage
\bibliographystyle{unsrtnat}
\bibliography{jasek}

\appendix
\input{text/dvd.tex}

\end{document}
